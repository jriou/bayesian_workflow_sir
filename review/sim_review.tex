\documentclass[12pt]{article}
\usepackage[utf8]{inputenc}
\usepackage[top=2cm, bottom=2cm, left=2cm, right=2cm]{geometry}
\usepackage{color}

\newcounter{question}
\newcommand{\name}{00}

\renewcommand{\thefigure}{R\arabic{figure}}
\renewcommand{\thetable}{R\arabic{table}}

\newcommand{\question}[1]{\stepcounter{question} \noindent \textbf{Comment \name.\thequestion} \emph{#1} }

\newcommand{\answer}[1]{\noindent \textbf{Answer to \name.\thequestion} #1 \mbox{}\\}

\newcommand{\newperson}[2]{\renewcommand{\name}{#2} \setcounter{question}{0} \newpage \noindent \textbf{\Large Answers to #1} \\}

\begin{document}
	
	{\large \textbf{Response to the comments about
			the submitted paper \emph{``Bayesian workflow for disease transmission modeling in
				Stan''
		}}}
	
	\vspace{3em}
	
	
	We thank the reviewers for their constructive comments. We have
	addressed all of them and modified the paper accordingly. Our
	detailed answers follow. 
	Please note that reviewers' comments are in italics while our
	answers are not. 
	
	
	\newperson{Reviewer 1}{R1}
	
%	\question{The paper is topical, very well written, balanced and cites the important related literature. General Bayesian workflow has been discussed also elsewhere, but the use of the domain specific example makes presentation more approachable and well targeted for the readers of Statistics in Medicine. I would be happy to see this published as is with just one tiny minor comment the authors could fix without need for re-review.}
%	
%	\answer{}
	
	\question{The Rhat shown in the Stan diagnostics table on page 11 is version 4 of Rhat, while Gelman and Rubin presented the version 1. After Gelman and Rubin's first version there have been many improvements, and calling the shown Rhat as Gelman-Rubin ratio is misleading. The Rhat version shown in the table on page 11 is described in Gelman et al, 2013 (BDA3) and in Stan user guide.}
	
	\answer{We agree with reviewer 1, the correct reference for improved r-hat is Vehtari et al, 2020.}
	
	\newperson{Reviewer 2}{R2}
	
	\question{The authors evidently assume that the reader is familiar with both Bayesian statistical
		modeling and the Stan software environment. But it Sections 2.1 and 2.2., the authors
		present the Bayesian idea as if it is entirely new to the reader, going so far as to
		define the posterior distribution and the “proportional to” notation. But further in the
		manuscript (e.g., Section 4.3), the authors discuss setting up a Stan model as if the
		reader already knows the general procedure for doing this and how to interface it with,
		e.g., R. I would think that if one knows how to use Stan as well as the authors assume,
		then one almost certainly has a working knowledge of Bayesian statistics. Indeed,
		in Section 4.2, the authors very casually mention weakly informative priors without
		defining them or their use. This seems like a big leap from having to define the $\propto$
		notation just a few pages earlier. I think the paper could be strengthened perhaps by
		swapping things around here. It’s OK to assume the reader is familiar with Bayesian
		modeling, but maybe spend more time on how one installs and sets up Stan? This
		approach would be more likely to attract readers like myself who are quite familiar
		with Bayesian statistics but have extremely limited knowledge of Stan. In its current
		form, if one is not familiar with Stan, they will be completely lost.}
	
	\answer{Answer}
	
		\question{Similarly, as part of introducing Stan to the broader readership, a little more time
			could be spent convincing the reader why it’s worthwhile; i.e., why Hamiltonian Monte
			Carlo (HMC) works and under what circumstances it is likely to be effective (high-
			dimensional parameter spaces? Multi-modal posteriors? Disconnected support? etc.).
			Again, it strikes me as odd that the authors spend pretty much no time at all reviewing
			HMC, but then feel the need to define the Gelman-Rubin Rhat statistic and effective
			sample size.}
	
	\answer{Answer}
	
	\question{At times the paper feels more like an applied mathematics paper dealing with systems
		of ODEs for disease spread than a statistics paper for Bayesian modeling. While I
		think it’s fine to assume some familiarity with differential equations, perhaps a little
		more effort could be spent on discussing numerical solvers, particularly those that are
		employed in Stan. Much less effort, if any at all, needs to be spent on more tangential
		points. For instance, on my reading of the tutorial, I did not really see what Section
		5.3 added to the paper. It seems irrelevant to someone looking to do model building
		with Stan.}
	
	\answer{Answer}
	
	\question{In the last paragraph before Section 4.6.2 on page 10, the authors give the impression
		that difficulties with MCMC mixing are always due to the model, and never to the
		chosen algorithm. I understand what they are saying and it is a good point. At the
		same time, though, I do not believe in a one-size-fits-all approach to MCMC. My
		experience has been that quite often the algorithm itself should be carefully thought
		out, in addition to the model itself; e.g., marginalization, adaptive Metropolis, delayed
		rejection, ordinary Gibbs, etc., etc. all have their uses. HMC is certainly a powerful
		MCMC framework, but it’s not the solution to every problem. I don’t think the reader
		should be given an impression to the contrary.}
	
	\answer{Answer}
	
	\question{First paragraph of Section 5: I think “ripe” should be “reap.”}
	
	\answer{Answer}
	
	\bibliography{Mendeley}
	\bibliographystyle{ieeetr}
	
\end{document}